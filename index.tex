% Options for packages loaded elsewhere
% Options for packages loaded elsewhere
\PassOptionsToPackage{unicode}{hyperref}
\PassOptionsToPackage{hyphens}{url}
\PassOptionsToPackage{dvipsnames,svgnames,x11names}{xcolor}
%
\documentclass[
  letterpaper,
  DIV=11,
  numbers=noendperiod]{scrreprt}
\usepackage{xcolor}
\usepackage{amsmath,amssymb}
\setcounter{secnumdepth}{5}
\usepackage{iftex}
\ifPDFTeX
  \usepackage[T1]{fontenc}
  \usepackage[utf8]{inputenc}
  \usepackage{textcomp} % provide euro and other symbols
\else % if luatex or xetex
  \usepackage{unicode-math} % this also loads fontspec
  \defaultfontfeatures{Scale=MatchLowercase}
  \defaultfontfeatures[\rmfamily]{Ligatures=TeX,Scale=1}
\fi
\usepackage{lmodern}
\ifPDFTeX\else
  % xetex/luatex font selection
\fi
% Use upquote if available, for straight quotes in verbatim environments
\IfFileExists{upquote.sty}{\usepackage{upquote}}{}
\IfFileExists{microtype.sty}{% use microtype if available
  \usepackage[]{microtype}
  \UseMicrotypeSet[protrusion]{basicmath} % disable protrusion for tt fonts
}{}
\makeatletter
\@ifundefined{KOMAClassName}{% if non-KOMA class
  \IfFileExists{parskip.sty}{%
    \usepackage{parskip}
  }{% else
    \setlength{\parindent}{0pt}
    \setlength{\parskip}{6pt plus 2pt minus 1pt}}
}{% if KOMA class
  \KOMAoptions{parskip=half}}
\makeatother
% Make \paragraph and \subparagraph free-standing
\makeatletter
\ifx\paragraph\undefined\else
  \let\oldparagraph\paragraph
  \renewcommand{\paragraph}{
    \@ifstar
      \xxxParagraphStar
      \xxxParagraphNoStar
  }
  \newcommand{\xxxParagraphStar}[1]{\oldparagraph*{#1}\mbox{}}
  \newcommand{\xxxParagraphNoStar}[1]{\oldparagraph{#1}\mbox{}}
\fi
\ifx\subparagraph\undefined\else
  \let\oldsubparagraph\subparagraph
  \renewcommand{\subparagraph}{
    \@ifstar
      \xxxSubParagraphStar
      \xxxSubParagraphNoStar
  }
  \newcommand{\xxxSubParagraphStar}[1]{\oldsubparagraph*{#1}\mbox{}}
  \newcommand{\xxxSubParagraphNoStar}[1]{\oldsubparagraph{#1}\mbox{}}
\fi
\makeatother


\usepackage{longtable,booktabs,array}
\usepackage{calc} % for calculating minipage widths
% Correct order of tables after \paragraph or \subparagraph
\usepackage{etoolbox}
\makeatletter
\patchcmd\longtable{\par}{\if@noskipsec\mbox{}\fi\par}{}{}
\makeatother
% Allow footnotes in longtable head/foot
\IfFileExists{footnotehyper.sty}{\usepackage{footnotehyper}}{\usepackage{footnote}}
\makesavenoteenv{longtable}
\usepackage{graphicx}
\makeatletter
\newsavebox\pandoc@box
\newcommand*\pandocbounded[1]{% scales image to fit in text height/width
  \sbox\pandoc@box{#1}%
  \Gscale@div\@tempa{\textheight}{\dimexpr\ht\pandoc@box+\dp\pandoc@box\relax}%
  \Gscale@div\@tempb{\linewidth}{\wd\pandoc@box}%
  \ifdim\@tempb\p@<\@tempa\p@\let\@tempa\@tempb\fi% select the smaller of both
  \ifdim\@tempa\p@<\p@\scalebox{\@tempa}{\usebox\pandoc@box}%
  \else\usebox{\pandoc@box}%
  \fi%
}
% Set default figure placement to htbp
\def\fps@figure{htbp}
\makeatother





\setlength{\emergencystretch}{3em} % prevent overfull lines

\providecommand{\tightlist}{%
  \setlength{\itemsep}{0pt}\setlength{\parskip}{0pt}}



 


\KOMAoption{captions}{tableheading}
\makeatletter
\@ifpackageloaded{bookmark}{}{\usepackage{bookmark}}
\makeatother
\makeatletter
\@ifpackageloaded{caption}{}{\usepackage{caption}}
\AtBeginDocument{%
\ifdefined\contentsname
  \renewcommand*\contentsname{Table of contents}
\else
  \newcommand\contentsname{Table of contents}
\fi
\ifdefined\listfigurename
  \renewcommand*\listfigurename{List of Figures}
\else
  \newcommand\listfigurename{List of Figures}
\fi
\ifdefined\listtablename
  \renewcommand*\listtablename{List of Tables}
\else
  \newcommand\listtablename{List of Tables}
\fi
\ifdefined\figurename
  \renewcommand*\figurename{Figure}
\else
  \newcommand\figurename{Figure}
\fi
\ifdefined\tablename
  \renewcommand*\tablename{Table}
\else
  \newcommand\tablename{Table}
\fi
}
\@ifpackageloaded{float}{}{\usepackage{float}}
\floatstyle{ruled}
\@ifundefined{c@chapter}{\newfloat{codelisting}{h}{lop}}{\newfloat{codelisting}{h}{lop}[chapter]}
\floatname{codelisting}{Listing}
\newcommand*\listoflistings{\listof{codelisting}{List of Listings}}
\makeatother
\makeatletter
\makeatother
\makeatletter
\@ifpackageloaded{caption}{}{\usepackage{caption}}
\@ifpackageloaded{subcaption}{}{\usepackage{subcaption}}
\makeatother
\usepackage{bookmark}
\IfFileExists{xurl.sty}{\usepackage{xurl}}{} % add URL line breaks if available
\urlstyle{same}
\hypersetup{
  pdftitle={Tiara Kusuma Wardhani},
  pdfauthor={Tiara Kusuma Wardhani},
  colorlinks=true,
  linkcolor={blue},
  filecolor={Maroon},
  citecolor={Blue},
  urlcolor={Blue},
  pdfcreator={LaTeX via pandoc}}


\title{Tiara Kusuma Wardhani}
\usepackage{etoolbox}
\makeatletter
\providecommand{\subtitle}[1]{% add subtitle to \maketitle
  \apptocmd{\@title}{\par {\large #1 \par}}{}{}
}
\makeatother
\subtitle{Portfolio Asesmen II-2100 KIPP}
\author{Tiara Kusuma Wardhani}
\date{2025-10-22}
\begin{document}
\maketitle

\renewcommand*\contentsname{Table of contents}
{
\hypersetup{linkcolor=}
\setcounter{tocdepth}{2}
\tableofcontents
}

\bookmarksetup{startatroot}

\chapter*{Hai, terima kasih sudah mampir!
🙋🏻‍♀️}\label{hai-terima-kasih-sudah-mampir}
\addcontentsline{toc}{chapter}{Hai, terima kasih sudah mampir! 🙋🏻‍♀️}

\markboth{Hai, terima kasih sudah mampir! 🙋🏻‍♀️}{Hai, terima kasih sudah
mampir! 🙋🏻‍♀️}

Saya \textbf{Tiara Kusuma Wardhani}, dan website ini adalah hasil dari
tugas mata kuliah \textbf{Komunikasi Interpersonal dan Publik (II2100)}
di Program Studi \textbf{Sistem dan Teknologi Informasi, STEI-ITB}.

Melalui proyek, tulisan, dan karya kecil yang ada di sini, saya mencoba
untuk mencari jawaban atas satu hal yang sederhana:\\
\textbf{Bagaimana cara kita bisa berkontribusi untuk orang lain tanpa
kehilangan siapa diri kita?}

Perjalanan dalam memahami komunikasi ini masih terus berlanjut, dan
mungkin tak akan pernah selesai. Namun, setiap langkah kecil, setiap
refleksi, percakapan, dan pertemuan merupakan bagian yang tak
terpisahkan dari proses tersebut. Mari terus belajar, bukan untuk
menjadi sempurna, tapi untuk lebih peduli dan memahami satu sama lain.

Selamat menjelajahi dan semoga bermanfaat! 😊

\bookmarksetup{startatroot}

\chapter{UTS-1 All About Me}\label{uts-1-all-about-me}

\textbf{All About Me}

Saya Tiara Kusuma Wardhani, anak tunggal yang tumbuh di kota yang sejuk
dan sendu, Malang. Menjadi anak tunggal adalah sesuatu yang istimewa
bagi saya, hingga saya bisa sampai pada titik ini dengan seluruh
pengalaman yang saya bawa. Bagi saya, mengenal diri sendiri adalah
perjalanan yang tidak pernah selesai. Semakin saya tumbuh, semakin saya
sadar bahwa memahami siapa diri saya bukan sekadar soal kepribadian,
tetapi juga tentang bagaimana saya berinteraksi, beradaptasi, dan
membangun hubungan yang bermakna dengan orang lain.

Kalau saya harus mendeskripsikan diri saya dengan tiga kata, mungkin
kata-kata itu adalah terstruktur, empatik, dan tegas. Saya termasuk
orang yang suka merencanakan sesuatu dengan matang. Setiap langkah yang
saya ambil biasanya sudah memiliki tujuan yang jelas, karena saya
percaya bahwa hasil yang baik lahir dari proses yang terencana. Namun,
di balik sifat logis dan sistematis itu, saya juga punya sisi empatik
yang kuat. Saya mudah memahami perasaan orang lain dan sering kali tahu
kapan seseorang sedang tidak baik-baik saja, bahkan sebelum mereka
menceritakannya. Dan seperti hasil Color Personality Test saya, saya
punya sisi Red, sisi yang berorientasi pada hasil, ingin segalanya
berjalan efisien, dan tidak mudah puas sebelum target tercapai.

Namun, karakter seperti ini tidak terbentuk dalam semalam. Ketika saya
masih SD, saya pernah merasa tidak begitu disukai oleh teman-teman saya.
Mereka tetap mau berada di sekitar saya, tapi saya tahu, itu lebih
karena saya pintar dan sering membantu mereka, bukan karena mereka
benar-benar mengenal saya. Di masa itu, saya mulai sadar bahwa
kecerdasan saja tidak cukup untuk membuat hubungan yang hangat. Ada hal
lain yang lebih penting: cara kita memahami dan mengekspresikan diri.

Saat masuk SMP, hidup saya berubah pelan-pelan. Saya bertemu dengan
lingkungan baru, teman-teman baru, dan situasi sosial yang menantang
saya untuk beradaptasi. Di masa itu, saya belajar bagaimana caranya
menunjukkan diri saya dengan cara yang lebih sehat dan otentik. Saya
mulai belajar bahwa tidak semua hal perlu diatur, bahwa komunikasi bukan
tentang siapa yang paling pintar bicara, tetapi siapa yang paling tulus
mendengarkan. Dari situ, saya mulai mengelola kepribadian saya,
bagaimana bersikap tegas tanpa membuat orang lain merasa tertekan,
bagaimana jujur tanpa harus terlihat keras.

Hubungan antar manusia, bagi saya, adalah hal yang unik dan menantang.
Setiap orang memiliki kepribadiannya masing-masing, dan justru di
situlah letak keindahannya. Tidak semua orang bisa cocok dengan cepat,
tapi kalau kita mau beradaptasi, pasti akan ada titik temu. Hubungan
saya dengan orang tua adalah hubungan yang paling spesial. Dari mereka,
saya belajar tentang arti kasih sayang yang tidak bersyarat dan tentang
komunikasi yang sederhana tapi bermakna. Saya tumbuh di lingkungan yang
mengajarkan bahwa perhatian tidak selalu ditunjukkan lewat kata-kata,
kadang justru lewat tindakan kecil yang konsisten, seperti cara ibu
menyiapkan makanan tanpa diminta, atau ayah yang diam-diam memastikan
semua kebutuhan kami terpenuhi.

Pemahaman tentang orang lain semakin dalam ketika saya duduk di bangku
SMA. Saat itu saya bertemu dengan banyak teman yang memiliki ego dan
ambisi yang sama besar dengan saya. Jujur, awalnya sering bentrok. Tapi
dari sanalah saya belajar bahwa setiap orang membawa cerita dan latar
belakang yang membentuk cara berpikir mereka. Saya belajar menahan diri,
mencoba memahami bukan untuk membenarkan, tapi untuk menemukan
keseimbangan. Itulah titik di mana empati saya mulai tumbuh bukan hanya
sebagai reaksi emosional, tapi sebagai sikap hidup.

Dalam berinteraksi, saya menjunjung tinggi nilai kepercayaan dan kasih
sayang yang membawa kebaikan. Saya percaya bahwa hubungan yang tulus
hanya bisa tumbuh kalau ada rasa percaya. Kepercayaan adalah fondasi
dari setiap bentuk komunikasi interpersonal yang sehat. Saya selalu
berusaha membangun hubungan yang berlandaskan saling menghargai, dan
menebarkan energi positif di setiap kesempatan. Mungkin karena itu saya
sering dijuluki teman-teman sebagai ``si paling positive vibes'' di
fakultas. Saya tidak menganggap itu pujian semata, tapi pengingat bahwa
sikap saya bisa memengaruhi suasana orang lain.

Meski begitu, saya bukan orang yang sempurna. Ada hal-hal yang masih
saya perjuangkan dalam diri saya sendiri. Salah satunya adalah
kecenderungan untuk menunda-nunda sesuatu yang sudah direncanakan.
Lucunya, ini agak bertentangan dengan sifat terstruktur saya. Kadang
saya punya rencana besar dan matang, tapi menundanya terlalu lama sampai
akhirnya saya kecewa pada diri sendiri. Namun, saya belajar untuk tidak
terlalu keras pada diri sendiri. Saya berusaha menjaga ritme saya,
menerima bahwa produktivitas juga punya siklusnya. Yang penting bukan
seberapa cepat saya bergerak, tapi seberapa konsisten saya melangkah.

Dalam menghadapi konflik, saya lebih memilih untuk tenang dan mencari
akar masalahnya. Saya percaya setiap konflik punya sebab yang bisa
diurai kalau kita mau mendengar dengan kepala dingin. Saya tidak suka
memperpanjang perdebatan, karena menurut saya energi lebih baik dipakai
untuk mencari solusi. Konflik memang tidak bisa dihindari, tapi bisa
dikelola dengan cara yang lebih dewasa dan konstruktif.

Dalam beberapa tahun terakhir, saya belajar tentang memimpin diri
sendiri. Saya mulai mengenali siapa saya, apa yang saya inginkan, apa
yang memotivasi saya, dan apa yang menjadi batas kemampuan saya. Saya
menyadari bahwa kepemimpinan bukan hanya soal mengatur orang lain, tapi
juga tentang mengarahkan diri agar tetap berjalan di jalur yang benar,
bahkan ketika tidak ada yang melihat.

Ke depan, saya ingin dikenal sebagai seseorang yang memimpin dengan
tegas, tapi tetap penuh empati. Saya ingin bisa menciptakan lingkungan
di mana semua orang merasa dihargai, bukan karena saya baik, tapi karena
saya benar-benar mendengarkan mereka. Saya ingin menjadi pribadi yang
memancarkan semangat positif, seseorang yang bisa menenangkan sekaligus
menggerakkan.

Kalau ada satu pesan yang ingin saya bagikan untuk siapa pun yang
membaca ini, mungkin pesannya sederhana:

``Setiap orang memiliki kecepatannya masing-masing. Jika kamu tidak bisa
berlari, berjalanlah, karena kita semua unik.''

Saya percaya hidup bukan perlombaan untuk siapa yang lebih cepat sampai,
tapi tentang bagaimana kita menikmati proses menuju tujuan
masing-masing. Dan selama saya bisa terus belajar, memahami, dan berbuat
baik, saya yakin saya sedang berada di jalur yang benar, jalur yang
membawa saya menjadi versi terbaik dari diri saya sendiri.

\bookmarksetup{startatroot}

\chapter{UTS-2 My Songs for You}\label{uts-2-my-songs-for-you}

The Fire You Gave

Music: SUNO

Audio: \href{./The\%20Fire\%20You\%20Gave.mp3}{The Fire You Gave}

Penjelasan Lagu ``The Fire You Gave''

Lagu \textbf{``The Fire You Gave''} saya tulis sebagai bentuk komunikasi
pribadi untuk orang tua saya, yang selalu menjadi sumber kekuatan dan
nilai dalam hidup saya. Dalam lagu ini, saya menggunakan campuran bahasa
Indonesia dan Inggris untuk menggambarkan keseimbangan antara dunia
tradisional keluarga dengan dunia modern tempat saya tumbuh. Lagu ini
menjadi ungkapan rasa syukur, rindu, dan tekad untuk terus berjuang,
sekaligus mencerminkan komunikasi interpersonal yang empatik dengan
menyampaikan cinta, pengakuan, dan penghargaan kepada orang tua secara
jujur.

\textbf{Verse 1} Lirik pertama menggambarkan kenangan saya bersama orang
tua, di mana kami duduk bersama, berbicara tanpa batas dan rasa takut,
membahas mimpi dan dunia yang luas. Mereka mengingatkan saya untuk tidak
takut untuk mengejar impian, memberikan dorongan yang tidak hanya
mendalam tetapi juga penuh keyakinan.

\begin{quote}
\textbf{``That night I still remember so clear,\\
We sat talking, no boundaries, no fear,\\
About dreams, about the world so wide,\\
You said, `My child, don't be afraid to reach far and wide.'\,''}
\end{quote}

\textbf{Verse 2} Verse kedua menceritakan pengorbanan orang tua, dengan
simbol ``keringat yang menjadi doa''. Mereka telah berusaha keras untuk
menciptakan jalan bagi saya, dengan mengorbankan banyak hal agar saya
bisa berdiri dan mencapai posisi saya sekarang. Ini mencerminkan betapa
besar perjuangan orang tua saya dalam hidup saya.

\begin{quote}
\textbf{``Your sweat turned into prayers,\\
Your steps crossed islands, passed through time and air,\\
You paved the road with your own hands,\\
So I could stand where I am today, where I stand.''}
\end{quote}

\textbf{Chorus} Chorus mengungkapkan rasa terima kasih saya kepada orang
tua, meskipun kami terpisah jarak, suara mereka tetap bersinar seperti
bintang utara, menjadi petunjuk dan sumber keberanian saya. Mereka
memberi saya ``api'' yang terus menyala dalam hati saya, yang memberi
saya kekuatan untuk terus berjuang.

\begin{quote}
\textbf{``Father, Mother, hear me from afar,\\
Your voices still shine like northern stars,\\
I'm fighting here with the fire you gave,\\
A flame you lit deep in my heart, to be brave.''}
\end{quote}

\textbf{Verse 3} Verse ketiga menggambarkan betapa kuatnya keteguhan
hati orang tua saya, yang selalu penuh kebanggaan dan tidak pernah ragu,
meskipun saya sekarang jauh dari rumah. Kehangatan mereka tetap memberi
arah dalam hidup saya, meskipun kami terpisah oleh jarak.

\begin{quote}
\textbf{``God is the witness to your every stride,\\
Never tired, never doubting, full of pride,\\
Now that I'm far from home's embrace,\\
Your warmth still guides me, wherever I chase.''}
\end{quote}

\textbf{Bridge} Bagian bridge menambahkan unsur kerinduan dan kesepian
yang sering datang. Namun, cinta orang tua saya tetap menjadi kekuatan
yang menjaga jiwa saya, mengingatkan saya pada kenangan indah saat
bersama mereka.

\begin{quote}
\textbf{``Sometimes loneliness comes knocking on my door,\\
But your love keeps my soul evermore,\\
I miss our laughter, those peaceful nights,\\
When stories healed, and prayers held us tight.''}
\end{quote}

\textbf{Final Chorus} Chorus terakhir merupakan ungkapan rasa terima
kasih saya yang mendalam kepada orang tua, mengakui bahwa menjadi anak
mereka adalah anugerah terbesar dalam hidup saya. Saya berjanji untuk
terus melangkah maju dengan membawa ``api'' yang mereka beri, sebagai
penerang jalan hidup saya.

\begin{quote}
\textbf{``Father, Mother, I'm grateful, truly I am,\\
To be your child is life's greatest plan,\\
I'll walk ahead and proudly say,\\
I'll keep the fire---you gave---that lights my way.''}
\end{quote}

\textbf{Kesimpulan} Lagu ini adalah simbol dari hubungan saya dengan
orang tua---perjuangan, doa, dan cinta tanpa batas yang mereka berikan.
``The Fire You Gave'' bukan hanya tentang mengingat apa yang telah
diberikan oleh orang tua, tetapi juga tentang bagaimana saya membawa
semangat itu dalam perjalanan hidup saya, menjaga api tersebut tetap
menyala, dan terus berkembang menjadi versi terbaik dari diri saya.

\bookmarksetup{startatroot}

\chapter{UTS-3 My Stories for You}\label{uts-3-my-stories-for-you}

Kadang, hidup terasa seperti panggung besar yang menuntut kita untuk
tahu siapa diri kita bahkan sebelum kita sempat mengenalnya. Saya masih
ingat masa kecil saya di SD, masa di mana dunia terasa sederhana, tapi
hati saya justru rumit. Saya bukan anak yang disukai banyak orang. Entah
kenapa, ada saja yang tidak cocok dengan cara saya berbicara, cara saya
bergaul, atau mungkin hanya karena saya terlalu ``berbeda.'' Lucunya,
meski begitu, mereka tetap mau berteman dengan saya, karena, yah\ldots{}
saya pintar. Mungkin itu satu-satunya alasan mereka bertahan di sekitar
saya, dan jujur, waktu itu saya menerimanya begitu saja. Saya pikir,
tidak apa-apa kalau orang tidak menyukai saya, asal mereka masih mau
bersama saya. Tapi sekarang saya tahu, itu bukan cara yang sehat untuk
memandang diri sendiri.

Saat masa itu berakhir, dan SMP mulai membuka gerbangnya, saya melihat
kesempatan untuk berubah. Tidak ada seorang pun teman SD saya yang masuk
ke SMP yang sama. Saya benar-benar sendirian, tapi kali ini, saya justru
merasa itu bukan kutukan, melainkan awal yang baru. Saya memilih SMP
terbaik di kota Malang, dengan alasan yang mungkin sedikit ambisius:
saya ingin menjadi bintang di antara para bintang. Orang tua saya sempat
beberapa kali bertanya, ``Yakin, Nak? SMP itu berat, saingannya kuat.''
Tapi saya tetap bersikeras. Saya ingin membuktikan sesuatu, bahwa saya
bisa bersinar bukan karena kebetulan, tapi karena saya memang pantas.

Saya masih ingat hari pertama di sekolah baru itu. Semua terlihat luar
biasa, gedungnya megah, murid-muridnya pintar, dan saya\ldots{} saya
cuma berusaha tidak salah jalan ke toilet. Tapi di balik semua kegugupan
itu, saya punya tekad yang besar. Saya berkata pada diri sendiri, kalau
dunia ini panggung, maka saya akan belajar memainkan peran saya dengan
lebih baik daripada sebelumnya.

Hari demi hari berlalu, dan perlahan, saya mulai menemukan ritme saya.
Saya belajar mengekspresikan diri, bukan dengan memaksakan orang untuk
menyukai saya, tapi dengan menjadi diri saya yang tulus. Rasanya
seperti\ldots{} menyalakan lampu kecil di ruangan yang dulu gelap. Dan
ternyata, banyak orang yang datang mendekat karena mereka melihat cahaya
itu.

Perjalanan saya tidak berhenti di sana. SMA datang dengan tantangan
baru, lebih besar, lebih ramai, tapi juga lebih seru. Saya tetap
bersinar, tapi kali ini saya belajar bahwa menjadi bintang bukan berarti
bersaing dengan cahaya lain. Kadang, justru indah saat kita bisa
bersinar bersama-sama.

Lalu, hidup membawa saya ke babak berikutnya: kuliah. Saya diterima di
ITB, di fakultas terbaik, dikelilingi oleh orang-orang hebat. Di
sinilah, untuk pertama kalinya setelah sekian lama, saya merasa kecil
lagi. Semua orang di sekitar saya tampak cemerlang, mereka cepat,
pandai, kreatif, dan ambisius. Sementara saya? Saya sibuk menenangkan
diri agar tidak panik setiap kali tugas datang bersamaan dengan tiga
deadline dan satu ujian. Kadang saya bercanda ke teman dekat saya,
``Mungkin Tuhan sedang ngasih ujian, biar aku tahu rasanya jadi manusia
biasa.'' Tapi dalam tawa itu, ada sedikit kejujuran. Tekanan itu nyata.
Emosi saya sering naik turun, dan saya sempat kehilangan arah. Rasanya
seperti kembali menjadi anak SD, tidak tahu bagaimana harus menempatkan
diri, hanya saja kali ini, saya bukan takut tidak disukai, tapi takut
tidak cukup baik.

Di tengah semua kebingungan itu, saya mulai mencari kembali pegangan
yang dulu membuat saya kuat. Saya menemukan jawabannya bukan di buku,
bukan di catatan kuliah, tapi di telepon malam hari dengan orang tua
saya. Saya masih ingat setiap kali saya bercerita kepada mereka tentang
tugas yang menumpuk, tentang rasa takut saya gagal, atau tentang
hari-hari di mana saya hanya ingin berhenti sebentar. Ibu akan berkata
lembut, ``Tidak apa-apa lelah, Nak. Tapi jangan berhenti.'' Ayah
biasanya hanya tertawa kecil, lalu berkata, ``Ingat, perjuanganmu bukan
cuma milikmu sendiri.''

Mereka benar. Setiap kali saya hampir menyerah, saya selalu ingat
perjuangan mereka, bagaimana ayah saya rela bekerja jauh dari rumah,
menempuh perjalanan panjang setiap minggu hanya untuk memastikan
keluarga kecil kami baik-baik saja. Bagaimana ibu saya bangun setiap
malam, berdoa dalam sunyi di sepertiga malam terakhir, memohon agar saya
kuat dan tetap dijaga Tuhan. Saya tahu doa itu yang sampai pada saya,
dalam bentuk kekuatan yang entah datang dari mana setiap kali saya
hampir runtuh.

Saya sering berpikir, mungkin inilah cara cinta bekerja. Ia tidak selalu
terlihat megah atau dramatis. Kadang cinta hadir dalam bentuk sederhana:
sepotong nasihat, secangkir teh yang diseduh diam-diam, atau doa yang
tidak pernah disebutkan tapi terus dikirimkan. Dan mungkin, ini juga
cara kasih sayang orang tua bekerja, diam, tapi dalam.

Lucunya, meski saya sering terlalu serius memikirkan masa depan, saya
juga tahu saya bukan orang yang selalu tegar. Ada hari-hari di mana saya
merasa ingin menyerah. Ada malam-malam di mana saya ingin berteriak,
``Sudah, cukup!'' Tapi kemudian saya ingat, mungkin di saat yang sama,
ibu saya sedang menengadahkan tangan, menyebut nama saya dalam doanya.
Lalu saya tertawa kecil sendiri dan berkata dalam hati, ``Ya sudah, masa
saya mau nyerah duluan dari doa ibu?''

Sekarang, ketika saya melihat perjalanan saya ke belakang, saya sadar
betapa banyak hal yang telah berubah. Saya bukan lagi anak SD yang
mencari pengakuan, bukan juga siswa SMP yang hanya ingin bersinar. Saya
sekarang adalah seseorang yang belajar untuk mencintai proses, bukan
hanya hasilnya. Saya belajar bahwa dalam setiap perjuangan, Tuhan selalu
menyiapkan hal-hal kecil yang membuat kita bertahan, entah itu senyum
teman, doa orang tua, atau keberanian yang tiba-tiba muncul entah dari
mana.

Yang paling saya syukuri adalah, saya tidak lagi ingin menjadi ``bintang
di antara bintang.'' Saya hanya ingin menjadi cahaya kecil yang bisa
memberi terang, walau sedikit. Dan kalau ada satu hal yang ingin saya
katakan untuk diri saya yang dulu, anak kecil yang dulu merasa tidak
disukai, saya ingin berkata, ``Kamu tidak perlu berubah untuk diterima.
Cukup terus tumbuh, dan Tuhan akan menempatkanmu di tempat yang tepat.''

Kini, saya menjalani hidup dengan lebih tenang. Saya masih punya banyak
mimpi, tentu saja. Saya masih sering cemas, masih sering kelelahan, dan
ya, kadang masih menunda tugas (itu penyakit lama yang belum sembuh).
Tapi setiap kali saya hampir kehilangan arah, saya selalu ingat satu hal
sederhana: Bahwa setiap langkah saya, sekecil apa pun, adalah bentuk
cinta dan doa yang sedang berjalan.

Dan bagi saya, itu sudah cukup untuk terus melangkah, pelan-pelan, tapi
pasti.

\bookmarksetup{startatroot}

\chapter{UTS-4 My SHAPE (Spiritual Gifts, Heart, Abilities, Personality,
Experiences)}\label{uts-4-my-shape-spiritual-gifts-heart-abilities-personality-experiences}

\textbf{Sumber} \href{StrengthsProfile-Tiara-Kusuma.pdf}{VIA assessment}

\begin{center}\rule{0.5\linewidth}{0.5pt}\end{center}

\section{Ringkasan 1 Halaman}\label{ringkasan-1-halaman}

\textbf{Peran Inti:}\\
Saya adalah seseorang yang berusaha terus berkembang dalam belajar dan
berkontribusi kepada orang-orang di sekitar saya. Sebagai anak tunggal
yang tumbuh di Malang, saya belajar untuk beradaptasi dengan lingkungan
dan menjadi lebih terbuka terhadap perubahan. Saya ingin menggunakan
pengalaman hidup dan kekuatan saya untuk memberi dampak positif di
kehidupan pribadi, keluarga, dan komunitas saya.

\textbf{Misi:}\\
Membantu orang lain berkembang, baik dalam hal pengetahuan maupun
pengembangan diri, dengan berbagi pengalaman dan empati. Saya ingin
menciptakan lingkungan yang mendukung dan positif, tempat di mana orang
merasa didengarkan dan dihargai.

\textbf{Kekuatan Utama:}\\
- Empati yang kuat, kemampuan untuk mendengarkan, dan memberikan
dukungan kepada orang lain. - Keinginan besar untuk terus belajar, tidak
hanya untuk diri sendiri, tetapi juga untuk berbagi ilmu dan pengalaman
dengan orang lain. - Keterampilan dalam beradaptasi dengan berbagai
situasi dan orang, serta membangun hubungan yang bermakna.

\textbf{Dampak yang Dituju:}\\
Menciptakan hubungan yang lebih baik dengan orang lain dan membantu
mereka berkembang. Saya ingin berbagi energi positif, memberikan
semangat, dan menciptakan dampak yang berarti di lingkungan sekitar.

\begin{center}\rule{0.5\linewidth}{0.5pt}\end{center}

\subsection{Peta SHAPE (singkat):}\label{peta-shape-singkat}

\begin{itemize}
\item
  \textbf{S --- Spiritual Gifts:}\\
  Empati, Pengertian, Memberikan dukungan, Kepemimpinan.
\item
  \textbf{H --- Heart (Minat \& Cinta Pelayanan):}\\
  Berbagi pengalaman hidup dan pengetahuan, mendukung teman-teman dan
  keluarga, membantu orang lain untuk lebih memahami diri mereka.
\item
  \textbf{A --- Abilities (Kemampuan):}\\
  Mendengarkan dengan empati, berbicara dengan percaya diri, memberikan
  motivasi, menjaga hubungan yang baik.
\item
  \textbf{P --- Personality (Gaya Kepribadian Kerja):}\\
  Terstruktur, reflektif, empatik, kreatif, kolaboratif, tenang dalam
  menghadapi tantangan.
\item
  \textbf{E --- Experiences (Pengalaman Kunci):}\\
  Pengalaman dalam mengatasi tantangan pribadi, pengalaman dalam
  memberikan dukungan kepada teman dan keluarga, serta belajar dari
  pengalaman yang saya jalani.
\end{itemize}

\begin{center}\rule{0.5\linewidth}{0.5pt}\end{center}

\section{S --- Spiritual Gifts (Karunia
Rohani)}\label{s-spiritual-gifts-karunia-rohani}

\textbf{Hasil Analisis VIA saya:}\\
- \textbf{Empati \& Pengertian:}\\
Saya memiliki kemampuan untuk merasakan dan memahami perasaan orang
lain, memberikan dukungan emosional ketika mereka membutuhkan. Ini
adalah kekuatan utama saya dalam berhubungan dengan orang lain.

\begin{itemize}
\tightlist
\item
  \textbf{Kepemimpinan yang Lembut:}\\
  Saya bisa memimpin dengan memberikan contoh yang baik dan mendukung
  orang lain untuk bertumbuh. Bukan kepemimpinan yang keras, tetapi
  lebih kepada bagaimana membantu orang lain mencapai potensi terbaik
  mereka.
\end{itemize}

\textbf{Indikator Bukti:}\\
- Dukungan yang saya berikan kepada teman-teman saat mereka membutuhkan
bimbingan. - Kemampuan saya untuk memotivasi dan membantu orang lain
mencapai tujuan mereka, baik dalam tugas atau kehidupan pribadi mereka.

\begin{center}\rule{0.5\linewidth}{0.5pt}\end{center}

\section{H --- Heart (Minat, Nilai,
Kepedulian)}\label{h-heart-minat-nilai-kepedulian}

\textbf{Hasil Analisis VIA saya:}\\
- \textbf{Pendidikan yang Bermakna:}\\
Saya sangat tertarik untuk memberikan pengalaman dan pengetahuan yang
bermanfaat bagi orang lain. Bagi saya, berbagi ilmu dan pengalaman hidup
adalah cara saya untuk memberi dampak positif.

\begin{itemize}
\item
  \textbf{Kesejahteraan Keluarga dan Teman:}\\
  Saya peduli dengan keluarga dan teman-teman saya, dan selalu berusaha
  memberikan dukungan emosional serta membantu mereka dalam menjalani
  kehidupan yang lebih baik.
\item
  \textbf{Menulis dan Berbagi Pengalaman:}\\
  Saya merasa menulis adalah cara terbaik untuk berbagi pengalaman dan
  memberikan inspirasi. Saya ingin menulis lebih banyak untuk memberi
  motivasi kepada orang lain.
\end{itemize}

\textbf{Masalah yang ingin dipecahkan:}\\
- Membantu orang lain untuk lebih mengenal dan menerima diri mereka
sendiri. - Menciptakan hubungan yang lebih bermakna dan tidak sekadar
hubungan biasa.

\begin{center}\rule{0.5\linewidth}{0.5pt}\end{center}

\section{A --- Abilities (Kemampuan
Andal)}\label{a-abilities-kemampuan-andal}

\textbf{Hasil Analisis VIA saya:}\\
- \textbf{Kemampuan Mendengarkan dan Memberi Dukungan:}\\
Saya bisa menjadi pendengar yang baik, dan saya merasa puas ketika bisa
memberi dukungan yang tepat kepada orang lain.

\begin{itemize}
\item
  \textbf{Berbicara di Depan Umum:}\\
  Saya merasa percaya diri berbicara di depan orang banyak, dan
  menggunakan kemampuan ini untuk memberi motivasi dan berbagi
  pengalaman.
\item
  \textbf{Membangun Hubungan yang Bermakna:}\\
  Saya percaya pada kekuatan hubungan yang saling mendukung. Saya
  terbuka untuk belajar dari orang lain, dan ingin selalu membantu orang
  lain merasa didengar.
\end{itemize}

\textbf{Indikator Bukti:}\\
- Pengalaman dalam memberi saran dan bimbingan kepada teman-teman saya.
- Keterlibatan dalam berbagai kegiatan sosial yang memungkinkan saya
untuk berinteraksi dan bekerja sama dengan berbagai orang.

\begin{center}\rule{0.5\linewidth}{0.5pt}\end{center}

\section{P --- Personality (Gaya Kerja \&
Kolaborasi)}\label{p-personality-gaya-kerja-kolaborasi}

\textbf{Hasil Analisis VIA saya:}\\
- \textbf{Terstruktur \& Reflektif:}\\
Saya suka merencanakan sesuatu dengan matang, dan saya juga merenung
untuk mengevaluasi diri dan memperbaiki cara kerja saya.

\begin{itemize}
\item
  \textbf{Empatik \& Kolaboratif:}\\
  Saya sangat memperhatikan perasaan orang lain, dan selalu berusaha
  bekerja sama untuk mencapai tujuan bersama. Saya percaya bahwa
  keberhasilan dicapai lebih mudah dengan saling mendukung.
\item
  \textbf{Kreatif \& Tenang dalam Krisis:}\\
  Saya memiliki kreativitas yang tinggi dalam menyelesaikan masalah,
  serta tetap tenang dan fokus ketika menghadapi krisis atau situasi
  yang penuh tekanan.
\end{itemize}

\begin{center}\rule{0.5\linewidth}{0.5pt}\end{center}

\section{E --- Experiences (Pengalaman
Pembentuk)}\label{e-experiences-pengalaman-pembentuk}

\textbf{Hasil Analisis VIA saya:}\\
- \textbf{Pengalaman Menghadapi Tantangan:}\\
Saya telah melewati banyak tantangan dalam hidup saya, yang membuat saya
lebih kuat dan lebih paham bagaimana cara menghadapi kesulitan.

\begin{itemize}
\tightlist
\item
  \textbf{Pengalaman Berbagi Pengalaman \& Memberi Dukungan:}\\
  Dalam perjalanan hidup saya, saya sering diminta untuk memberi
  dukungan kepada teman-teman yang sedang mengalami kesulitan.
\end{itemize}

\textbf{Pelajaran Inti:}\\
- Setiap tantangan adalah kesempatan untuk berkembang dan belajar lebih
banyak tentang diri sendiri dan orang lain. Saya belajar untuk tetap
sabar dan terbuka terhadap pembelajaran sepanjang hayat.

\begin{center}\rule{0.5\linewidth}{0.5pt}\end{center}

\section{Piagam Diri (Self‑Charter)}\label{piagam-diri-selfcharter}

\textbf{Misi Hidup:}\\
Menciptakan ruang yang mendukung pengembangan diri dan hubungan yang
bermakna, membantu orang lain tumbuh dengan memberi dukungan yang tulus.

\textbf{Nilai Inti:}\\
Empati, kejujuran, keberanian, kasih, kepercayaan.

\textbf{Peran Inti:}\\
Sebagai seseorang yang selalu mendukung orang lain untuk berkembang,
baik dalam hal pribadi maupun profesional, dan berusaha menciptakan
hubungan yang saling mendukung.

\textbf{Kompas Keputusan:}\\
(1) Dampak pada kesejahteraan orang lain, (2) Menjaga hubungan yang
sehat dan mendalam, (3) Keselarasan dengan nilai pribadi saya, (4)
Menciptakan dampak jangka panjang, (5) Terus belajar dan berkembang.

\textbf{Janji Pelayanan:}\\
Memberi dukungan dengan empati, mendengarkan dengan penuh perhatian, dan
membantu orang mencapai potensi terbaik mereka.

\textbf{Batasan:}\\
Menjaga keseimbangan antara pekerjaan dan kehidupan pribadi, serta
menolak bekerja dengan proyek atau orang yang tidak menghargai martabat
manusia.

\begin{center}\rule{0.5\linewidth}{0.5pt}\end{center}

\section{Narasi 90 Detik (Elevator
Pitch)}\label{narasi-90-detik-elevator-pitch}

``Saya Tiara Kusuma Wardhani, seorang mahasiswa yang berfokus pada
pendidikan dan pemberdayaan diri melalui pengalaman hidup dan komunikasi
yang empatik. Saya percaya bahwa berbagi pengalaman adalah cara terbaik
untuk membantu orang berkembang. Saya memiliki kekuatan dalam
mendengarkan, berbicara dengan percaya diri, dan memberikan dukungan
yang tepat. Saya ingin terus mengembangkan diri dan memberikan dampak
positif bagi orang-orang di sekitar saya.''

\begin{center}\rule{0.5\linewidth}{0.5pt}\end{center}

\section{Service‑Fit Map (Tempat Saya Paling
Berdampak)}\label{servicefit-map-tempat-saya-paling-berdampak}

Tempat saya paling berdampak adalah di \textbf{lingkungan kampus},
\textbf{keluarga}, dan \textbf{komunitas sosial}.\\
- \textbf{Kampus:} Sebagai mahasiswa yang aktif, saya ingin terus
memberikan kontribusi dengan membantu teman-teman dalam memahami materi
kuliah, membimbing mereka dalam proyek atau tugas, serta berkolaborasi
dalam kegiatan akademik atau sosial. Saya juga ingin mengembangkan
kurikulum yang relevan dengan kebutuhan nyata mahasiswa. -
\textbf{Keluarga:} Keluarga adalah tempat pertama saya belajar
nilai-nilai kehidupan. Saya selalu berusaha memberikan perhatian dan
dukungan emosional kepada keluarga saya, terutama orang tua, serta
berkontribusi dalam membangun hubungan yang lebih baik di lingkungan
rumah. - \textbf{Komunitas Sosial:} Saya aktif dalam berbagai kegiatan
sosial, membantu mereka yang membutuhkan bimbingan, terutama dalam
pendidikan dan pemberdayaan diri. Ini memberi saya kesempatan untuk
memberikan dampak nyata dalam kehidupan orang lain.

\begin{center}\rule{0.5\linewidth}{0.5pt}\end{center}

\section{Evidences (Artefak \& Tautan)}\label{evidences-artefak-tautan}

\begin{itemize}
\tightlist
\item
  \textbf{Karya Tulis:} Artikel yang saya tulis untuk berbagi pengalaman
  dan pengetahuan yang dapat menginspirasi orang lain.
\item
  \textbf{Presentasi \& Kuliah:} Pengalaman berbicara di depan umum,
  memberikan presentasi kepada teman-teman, serta pengalaman mengajar
  teman untuk memahami materi kuliah yang sulit.
\item
  \textbf{Kegiatan Sosial:} Kegiatan komunitas yang saya ikuti, di mana
  saya memberikan kontribusi untuk membantu orang lain melalui
  pendampingan dan edukasi.
\item
  \textbf{Proyek Kolaboratif:} Kerja tim dalam berbagai proyek yang
  menuntut saya untuk bekerja bersama orang dengan latar belakang
  berbeda dan menghasilkan sesuatu yang bermanfaat.
\end{itemize}

\begin{center}\rule{0.5\linewidth}{0.5pt}\end{center}

\section{Rencana Aksi 90 Hari (SMART)}\label{rencana-aksi-90-hari-smart}

\begin{itemize}
\tightlist
\item
  \textbf{Tugas Akademik:} Menyelesaikan semua tugas akademik tepat
  waktu, dengan fokus pada penyelesaian materi kuliah dan proyek
  kelompok.

  \begin{itemize}
  \tightlist
  \item
    \textbf{Outcome:} Semua tugas selesai dan dipresentasikan dengan
    baik.\\
  \item
    \textbf{Due:} T‑30 hari.
  \end{itemize}
\item
  \textbf{Mentoring Teman:} Menjadi mentor untuk minimal dua teman dalam
  menyelesaikan tugas atau pemahaman materi kuliah.

  \begin{itemize}
  \tightlist
  \item
    \textbf{Outcome:} Meningkatkan pemahaman teman-teman melalui diskusi
    dan bimbingan yang produktif.\\
  \item
    \textbf{Due:} T‑60 hari.
  \end{itemize}
\item
  \textbf{Kegiatan Sosial:} Berkontribusi dalam minimal satu kegiatan
  sosial atau komunitas untuk memberikan dampak positif di sekitar saya.

  \begin{itemize}
  \tightlist
  \item
    \textbf{Outcome:} Kegiatan yang memberikan manfaat langsung pada
    masyarakat sekitar, misalnya pendidikan atau pemberdayaan diri.\\
  \item
    \textbf{Due:} T‑90 hari.
  \end{itemize}
\end{itemize}

\begin{center}\rule{0.5\linewidth}{0.5pt}\end{center}

\section{SHAPE ↔ CPMK (Interpersonal \& Public
Communication)}\label{shape-cpmk-interpersonal-public-communication}

\begin{itemize}
\tightlist
\item
  \textbf{Self‑awareness \& refleksi (CPMK‑S):} Dalam \textbf{Piagam
  Diri} \& \textbf{Narasi 90 detik}, saya mencatat bagaimana cara saya
  memahami diri, terutama dalam merencanakan tujuan dan mendengarkan
  perasaan orang lain.
\item
  \textbf{Empati \& komunikasi etis (CPMK‑E):} Saya terus belajar untuk
  menjadi pendengar yang baik dan menjaga komunikasi yang etis dengan
  teman-teman serta orang yang saya ajak bekerja sama.
\item
  \textbf{Storytelling \& presentasi (CPMK‑P):} Kemampuan saya dalam
  berbicara di depan umum dan menulis lirik atau artikel menunjukkan
  keinginan untuk berbagi cerita yang dapat memberi inspirasi.
\item
  \textbf{Kolaborasi \& kepemimpinan (CPMK‑K):} Saya berusaha untuk
  bekerja sama dalam tim dengan empati dan memastikan kolaborasi yang
  saling mendukung untuk mencapai tujuan bersama.
\end{itemize}

\begin{center}\rule{0.5\linewidth}{0.5pt}\end{center}

\section{Self‑Assessment Rubrik UTS‑4}\label{selfassessment-rubrik-uts4}

\begin{longtable}[]{@{}
  >{\raggedright\arraybackslash}p{(\linewidth - 6\tabcolsep) * \real{0.3382}}
  >{\raggedright\arraybackslash}p{(\linewidth - 6\tabcolsep) * \real{0.4412}}
  >{\raggedleft\arraybackslash}p{(\linewidth - 6\tabcolsep) * \real{0.1471}}
  >{\raggedright\arraybackslash}p{(\linewidth - 6\tabcolsep) * \real{0.0735}}@{}}
\toprule\noalign{}
\begin{minipage}[b]{\linewidth}\raggedright
Kriteria
\end{minipage} & \begin{minipage}[b]{\linewidth}\raggedright
Deskripsi
\end{minipage} & \begin{minipage}[b]{\linewidth}\raggedleft
Skor (1--5)
\end{minipage} & \begin{minipage}[b]{\linewidth}\raggedright
Bukti
\end{minipage} \\
\midrule\noalign{}
\endhead
\bottomrule\noalign{}
\endlastfoot
\textbf{Kelengkapan SHAPE} & S‑H‑A‑P‑E jelas \& terisi & 5 & Semua
dimensi SHAPE (Spiritual Gifts, Heart, Abilities, Personality,
Experiences) telah diidentifikasi dengan jelas dan mendalam,
mencerminkan kepribadian dan kekuatan utama saya. \\
\textbf{Koherensi Piagam Diri} & misi‑nilai‑peran konsisten & 5 & Piagam
diri mencerminkan misi hidup yang konsisten untuk memberi dampak positif
melalui pendidikan dan pemberdayaan orang lain, sesuai dengan
nilai-nilai saya. \\
\textbf{Narasi 90 detik} & Ringkas, kuat, mengundang aksi & 5 & Narasi
90 detik saya jelas dan menarik, menggambarkan misi, kekuatan, dan
dampak yang ingin saya capai dalam kehidupan dan komunitas saya. \\
\textbf{Evidence \& Aksi 90 hari} & Tautan bukti \& rencana SMART & 4 &
Rencana aksi 90 hari saya mencakup tindakan SMART untuk mengembangkan
diri lebih lanjut, dengan bukti berupa pengalaman dan kegiatan saya yang
terkait dengan rencana tersebut. Namun, perlu lebih banyak bukti nyata
dari tindakan yang sudah dilakukan. \\
\end{longtable}

\textbf{Total (maks 20):} 19\\
\textbf{Tingkat:} ☐ A (≥85\%) ☐ B (70--84\%) ☐ C (60--69\%) ☐ D
(50--59\%) ☐ E (\textless50\%) ---

\section{Versi Ultra‑Ringkas (≤140
kata)}\label{versi-ultraringkas-140-kata}

``Saya Tiara Kusuma Wardhani, seorang mahasiswa yang berfokus pada
pemberdayaan orang lain melalui pendidikan berbasis nilai dan komunikasi
empatik. Karunia saya adalah mendengarkan, berbicara dengan percaya
diri, dan memberikan dukungan kepada orang lain untuk mencapai potensi
terbaik mereka. Saya berkomitmen untuk menciptakan hubungan yang
bermakna dan memberikan dampak positif melalui pendidikan dan
kolaborasi.''

\begin{center}\rule{0.5\linewidth}{0.5pt}\end{center}

\section{Piagam Diri --- Tiara Kusuma
Wardhani}\label{piagam-diri-tiara-kusuma-wardhani}

\textbf{Pernyataan Misi}\\
Saya berkomitmen untuk membantu orang lain berkembang, baik dalam
pengetahuan maupun pengembangan diri, dengan berbagi pengalaman dan
empati.

\textbf{S --- Signature Strengths (inti kekuatan khas)}\\
Empati, kemampuan berbicara di depan umum, mendengarkan dan memahami
orang lain.

\textbf{H --- Heart (nilai \& panggilan)}\\
Membantu orang lain berkembang, berbagi pengetahuan, menciptakan
hubungan yang saling mendukung.

\textbf{A --- Aptitudes \& Acquired Skills (bakat \& keterampilan
kunci)}\\
Menulis, berbicara di depan umum, mendengarkan dengan empati, kolaborasi
dalam tim.

\textbf{P --- Personality (gaya kerja yang menonjol)}\\
Terstruktur, empatik, kreatif, kolaboratif, tenang dalam menghadapi
tantangan.

\textbf{E --- Experiences (jejak pembentuk identitas)}\\
Pengalaman membantu teman memahami materi, menulis untuk berbagi
pengalaman, berkolaborasi dalam proyek sosial.

\textbf{Janji Praktis (Operating Principles)}\\
Berikan dampak positif dengan empati, kolaborasi, dan mendengarkan orang
lain dengan penuh perhatian.

\begin{center}\rule{0.5\linewidth}{0.5pt}\end{center}

\section{Narasi Diri (versi 90 detik)}\label{narasi-diri-versi-90-detik}

``Saya Tiara Kusuma Wardhani, seorang mahasiswa yang berfokus pada
pemberdayaan orang lain melalui pendidikan berbasis nilai dan komunikasi
yang empatik. Saya percaya bahwa setiap orang dapat berkembang dengan
dukungan yang tepat. Karunia saya adalah kemampuan untuk berbicara
dengan percaya diri dan memberikan dukungan yang tulus agar orang lain
bisa mencapai potensi terbaik mereka.''

\begin{center}\rule{0.5\linewidth}{0.5pt}\end{center}

\section{Narasi Diri (versi panjang, 3--5
paragraf)}\label{narasi-diri-versi-panjang-35-paragraf}

Saya Tiara Kusuma Wardhani, seorang mahasiswa yang terus berusaha
mengembangkan diri melalui pembelajaran dan berbagi pengalaman dengan
orang lain. Saya percaya bahwa setiap orang dapat berkembang apabila
diberi dukungan dan kesempatan yang tepat untuk belajar dan bertumbuh.
Saya senang membantu teman-teman dan keluarga melalui pendidikan dan
bimbingan yang saya berikan, karena saya percaya bahwa komunikasi yang
baik adalah kunci untuk membangun hubungan yang saling mendukung.

Saya juga menyadari pentingnya mendengarkan dan memahami perasaan orang
lain. Ini membuat saya lebih mudah beradaptasi dengan berbagai orang dan
situasi. Kekuatan utama saya adalah empati dan kemampuan untuk berbicara
di depan umum, yang memungkinkan saya untuk menginspirasi dan memberi
motivasi kepada orang lain.

Ke depan, saya ingin terus membantu orang berkembang dan menciptakan
lingkungan yang positif, baik di dalam kehidupan pribadi saya, keluarga,
maupun komunitas sosial saya. Saya ingin berkontribusi lebih banyak,
baik di dunia akademik maupun sosial, dengan berbagi pengetahuan,
memberi dukungan, dan membangun hubungan yang bermakna dengan orang di
sekitar saya.

\bookmarksetup{startatroot}

\chapter{UTS-5 My Personal Reviews}\label{uts-5-my-personal-reviews}

Hasil Self-Assessment UTS (URL: https://tiaraksmwr.github.io/)

\section{Identifikasi}\label{identifikasi}

\begin{itemize}
\tightlist
\item
  \textbf{Nama \& NIM penulis:} \textbf{Tiara Kusuma Wardhani --
  18224059}\\
\item
  \textbf{Penilai:} \textbf{Self-assessment (Tiara Kusuma Wardhani)}\\
\item
  \textbf{Catatan cakupan:} Halaman beranda memuat ``About Me'';
  navigasi ke ``My Songs for You'', ``My Stories for You'', ``My
  Shapes'', dan ``My Personal Reviews'' tersedia.
\end{itemize}

\begin{center}\rule{0.5\linewidth}{0.5pt}\end{center}

\section{Tinjauan Umum}\label{tinjauan-umum}

\begin{itemize}
\tightlist
\item
  \textbf{UTS-1 (All About Me):} Perkenalan diri sudah sangat padat dan
  menggambarkan siapa saya dengan baik. Namun, untuk meningkatkan
  keterlibatan pembaca, menambahkan elemen yang lebih personal dan
  sedikit humor akan lebih menarik.\\
\item
  \textbf{UTS-2 (My Songs for You):} Lagu sudah cukup baik, tetapi akan
  lebih lengkap dengan penambahan lirik dan penjelasan mengenai pesan di
  balik lagu tersebut agar lebih mendalam.\\
\item
  \textbf{UTS-3 (My Stories for You):} Cerita yang ditampilkan sangat
  kuat dan menginspirasi. Penulisan sudah sangat baik, hanya sedikit
  peningkatan alur cerita agar lebih menggugah.\\
\item
  \textbf{UTS-4 (My SHAPE):} Halaman sudah terisi dengan baik namun
  masih bisa lebih mendalam dalam menggambarkan pengalaman pribadi yang
  menonjolkan SHAPE.\\
\item
  \textbf{UTS-5 (My Personal Reviews):} Halaman sudah memiliki metode
  review yang baik, tetapi membutuhkan lebih banyak contoh nyata yang
  lebih konkret untuk memperdalam analisis dan refleksi.
\end{itemize}

\begin{center}\rule{0.5\linewidth}{0.5pt}\end{center}

\section{Tinjauan Spesifik + Skor
(1--5)}\label{tinjauan-spesifik-skor-15}

\subsection{UTS-1 --- All About Me}\label{uts-1-all-about-me-1}

\textbf{Skor per kriteria:} Orisinalitas \textbf{5}, Keterlibatan
\textbf{5}, Humor \textbf{4}, Wawasan/Insight \textbf{5} → \textbf{Total
19/20 (95\%)}

\textbf{Alasan singkat:} Penyampaian identitas diri sudah jelas dan
informatif. Namun, akan lebih menarik dengan sedikit sentuhan personal,
seperti cerita atau pengalaman yang lebih menggugah, serta humor agar
terasa lebih hidup.\\
\textbf{Saran perbaikan:} Masukkan anekdot pribadi yang lebih menarik
atau pengalaman hidup yang memberi pembaca kesempatan untuk terhubung
lebih dalam dengan diri kamu. Tambahkan sedikit humor atau refleksi
pribadi agar insight lebih mendalam.

\subsection{UTS-2 --- My Songs for You}\label{uts-2-my-songs-for-you-1}

\textbf{Skor per kriteria:} Orisinalitas \textbf{4}, Keterlibatan
\textbf{4}, Humor \textbf{3}, Inspirasi \textbf{4} → \textbf{Total 15/20
(85\%)}

\textbf{Alasan singkat:} Lagu yang kamu pilih memiliki pesan yang kuat
dan menginspirasi. Namun, untuk menambah dampak dan kedalaman, lirik dan
cerita di balik lagu perlu lebih diperjelas.\\
\textbf{Saran perbaikan:} Sertakan lirik lengkap dan sedikit cerita di
balik lagu. Jelaskan bagaimana lagu ini diciptakan dan apa pesan yang
ingin kamu sampaikan kepada pendengar.

\subsection{UTS-3 --- My Stories for
You}\label{uts-3-my-stories-for-you-1}

\textbf{Skor per kriteria:} Orisinalitas \textbf{5}, Keterlibatan
\textbf{5}, Pengembangan Narasi \textbf{4}, Inspirasi \textbf{5} →
\textbf{Total 19/20 (95\%)}

\textbf{Alasan singkat:} Cerita yang dibagikan sangat personal dan
emosional, dengan pesan yang sangat kuat. Beberapa cerita bisa lebih
terstruktur untuk meningkatkan alur narasi, tetapi secara keseluruhan
sudah sangat baik.\\
\textbf{Saran perbaikan:} Menambahkan pengantar yang lebih kuat di awal
cerita bisa membantu pembaca untuk lebih mudah terhubung dengan cerita.
Akhiri setiap cerita dengan refleksi yang memperjelas pesan yang ingin
disampaikan.

\subsection{UTS-4 --- My SHAPE}\label{uts-4-my-shape}

\textbf{Skor per kriteria:} Orisinalitas \textbf{4}, Keterlibatan
\textbf{5}, Pengembangan \textbf{5}, Inspirasi \textbf{4} →
\textbf{Total 18/20 (90\%)}

\textbf{Alasan singkat:} Halaman SHAPE sudah terisi dengan baik dan
menyajikan banyak informasi penting. Namun, beberapa pengalaman yang
lebih spesifik yang menggambarkan kekuatan kamu akan memberikan lebih
banyak konteks dan kedalaman.\\
\textbf{Saran perbaikan:} Tambahkan contoh lebih banyak mengenai
pengalaman yang menunjukkan kekuatan utama yang kamu miliki. Jelaskan
3-5 pengalaman yang bisa menggambarkan kekuatan SHAPE secara lebih
mendalam.

\subsection{UTS-5 --- My Personal
Reviews}\label{uts-5-my-personal-reviews-1}

\textbf{Skor per kriteria:} Pemahaman Konsep \textbf{5}, Analisis Kritis
\textbf{4}, Argumentasi \textbf{5}, Etos \& Empati \textbf{5},
Rekomendasi \textbf{4} → \textbf{Total 23/25 (92\%)}

\textbf{Alasan singkat:} Sudah ada pemahaman yang baik tentang metode
review. Penyampaian sangat jelas, namun membutuhkan lebih banyak contoh
yang konkret dan penjelasan lebih mendalam terkait etos dan empati dalam
karya yang di-review.\\
\textbf{Saran perbaikan:} Pilih satu karya dan buat review dengan
analisis yang lebih tajam. Sertakan kutipan yang relevan dan evaluasi
lebih dalam mengenai etos dan empati dalam karya tersebut.

\begin{center}\rule{0.5\linewidth}{0.5pt}\end{center}

\section{Rekap Skor (ringkas)}\label{rekap-skor-ringkas}

\begin{itemize}
\tightlist
\item
  \textbf{UTS-1:} 15/20 → \textbf{95\%}
\item
  \textbf{UTS-2:} 15/20 → \textbf{85\%}
\item
  \textbf{UTS-3:} 19/20 → \textbf{95\%}
\item
  \textbf{UTS-4:} 16/20 → \textbf{90\%}
\item
  \textbf{UTS-5:} 20/25 → \textbf{92\%}
\end{itemize}

\begin{center}\rule{0.5\linewidth}{0.5pt}\end{center}

\section{Langkah Perbaikan Cepat (prioritas 1
minggu)}\label{langkah-perbaikan-cepat-prioritas-1-minggu}

\begin{enumerate}
\def\labelenumi{\arabic{enumi}.}
\tightlist
\item
  \textbf{Lengkapi UTS-4 (My SHAPE)} dengan lebih banyak contoh nyata
  yang menunjukkan pengalaman yang menggambarkan kekuatan utama yang
  dimiliki.
\item
  \textbf{Tambahkan lirik + cerita di balik lagu di UTS-2} agar lebih
  memberi inspirasi dan pemahaman kepada pembaca/pendengar.
\item
  \textbf{Tulis 1 review pribadi penuh} di UTS-5, dengan lebih banyak
  analisis mendalam tentang karya yang dipilih.
\item
  \textbf{Revisi UTS-1} dengan memasukkan anekdot pribadi dan sedikit
  humor agar lebih menghubungkan pembaca dengan cerita kamu.
\end{enumerate}

\begin{center}\rule{0.5\linewidth}{0.5pt}\end{center}

\section{Peer Assessment}\label{peer-assessment}

\{\url{https://docs.google.com/spreadsheets/d/18iSaCOAKPnqslv58rd0lE4tUytbAnjAc/edit?usp=sharing&ouid=111341791953014744360&rtpof=true&sd=true}\}

\bookmarksetup{startatroot}

\chapter{UAS-1 My Concepts}\label{uas-1-my-concepts}

Mau hidup epik ? \href{lifestory.pdf}{Write your Life Story}

Apa itu berkonsep?

\url{https://youtu.be/QVfUlVBO80U?si=yM6q_rwV9rcDBbu7}

\bookmarksetup{startatroot}

\chapter{UAS-3 My Opinions}\label{uas-3-my-opinions}

SApa itu beropini? \href{BM\%20Opini.mp4}{Opini Berpengaruh}

Bagiamana menjaadi menarik? \href{./Interesting.mp4}{Menjadi Menarik}

\bookmarksetup{startatroot}

\chapter{UAS-3 My Innovations}\label{uas-3-my-innovations}

\bookmarksetup{startatroot}

\chapter{UAS-4 My Knowledge}\label{uas-4-my-knowledge}

Cara saya mengkomunikasikan sebuah pengatahuan sebagai petunjuk bagi
orang lain 1) saya tulis
\href{Rekomendasi\%20Presentasi\%20Efektif(Contoh\%20Makalah).pdf}{makalah
sebagai bahan utama} 2) lalu saya buat
\href{Contoh\%20Transkrip\%20Presentasi.pdf}{transkrip ucapan lisan} 3)
kemudian saya kembangkan
\href{Rekomendasi\%20Presentasi\%20(Contoh\%20Slides).pdf}{slide
pendukung trnsskrip} 4) lalu saya memproduksivideo audio visual
\url{https://youtu.be/ZbghfMvnPZc} \url{https://youtu.be/ZbghfMvnPZc}

\bookmarksetup{startatroot}

\chapter{UAS-5 My Professional
Reviews}\label{uas-5-my-professional-reviews}

Untuk melAkukan review, seperti pada
\href{../My_Personal_Reviews/Doc.5.Mengevaluasi-Esai-Berdasarkan-Rubrik.pdf}{pendekatan
AI}, kita membutuhkan rubrik

\bookmarksetup{startatroot}

\chapter{Summary}\label{summary}

In summary, this book has no content whatsoever.

\bookmarksetup{startatroot}

\chapter*{References}\label{references}
\addcontentsline{toc}{chapter}{References}

\markboth{References}{References}

\phantomsection\label{refs}




\end{document}
